%% Generated by Sphinx.
\def\sphinxdocclass{report}
\documentclass[letterpaper,10pt,polish]{sphinxmanual}
\ifdefined\pdfpxdimen
   \let\sphinxpxdimen\pdfpxdimen\else\newdimen\sphinxpxdimen
\fi \sphinxpxdimen=.75bp\relax
\ifdefined\pdfimageresolution
    \pdfimageresolution= \numexpr \dimexpr1in\relax/\sphinxpxdimen\relax
\fi
%% let collapsible pdf bookmarks panel have high depth per default
\PassOptionsToPackage{bookmarksdepth=5}{hyperref}

\PassOptionsToPackage{booktabs}{sphinx}
\PassOptionsToPackage{colorrows}{sphinx}

\PassOptionsToPackage{warn}{textcomp}
\usepackage[utf8]{inputenc}
\ifdefined\DeclareUnicodeCharacter
% support both utf8 and utf8x syntaxes
  \ifdefined\DeclareUnicodeCharacterAsOptional
    \def\sphinxDUC#1{\DeclareUnicodeCharacter{"#1}}
  \else
    \let\sphinxDUC\DeclareUnicodeCharacter
  \fi
  \sphinxDUC{00A0}{\nobreakspace}
  \sphinxDUC{2500}{\sphinxunichar{2500}}
  \sphinxDUC{2502}{\sphinxunichar{2502}}
  \sphinxDUC{2514}{\sphinxunichar{2514}}
  \sphinxDUC{251C}{\sphinxunichar{251C}}
  \sphinxDUC{2572}{\textbackslash}
\fi
\usepackage{cmap}
\usepackage[T1]{fontenc}
\usepackage{amsmath,amssymb,amstext}
\usepackage{babel}



\usepackage{tgtermes}
\usepackage{tgheros}
\renewcommand{\ttdefault}{txtt}



\usepackage[Sonny]{fncychap}
\ChNameVar{\Large\normalfont\sffamily}
\ChTitleVar{\Large\normalfont\sffamily}
\usepackage{sphinx}

\fvset{fontsize=auto}
\usepackage{geometry}


% Include hyperref last.
\usepackage{hyperref}
% Fix anchor placement for figures with captions.
\usepackage{hypcap}% it must be loaded after hyperref.
% Set up styles of URL: it should be placed after hyperref.
\urlstyle{same}

\addto\captionspolish{\renewcommand{\contentsname}{Spis Treści}}

\usepackage{sphinxmessages}
\setcounter{tocdepth}{2}



\title{piotr\_sreba\_290312}
\date{06 lis 2025}
\release{1.0.0}
\author{Piotr Sreba}
\newcommand{\sphinxlogo}{\vbox{}}
\renewcommand{\releasename}{Wydanie}
\makeindex
\begin{document}

\ifdefined\shorthandoff
  \ifnum\catcode`\=\string=\active\shorthandoff{=}\fi
  \ifnum\catcode`\"=\active\shorthandoff{"}\fi
\fi

\pagestyle{empty}
\sphinxmaketitle
\pagestyle{plain}
\sphinxtableofcontents
\pagestyle{normal}
\phantomsection\label{\detokenize{index::doc}}


\sphinxAtStartPar
Add your content using \sphinxcode{\sphinxupquote{reStructuredText}} syntax. See the
\sphinxhref{https://www.sphinx-doc.org/en/master/usage/restructuredtext/index.html}{reStructuredText}
documentation for details.

\sphinxAtStartPar
\sphinxstylestrong{Wstęp}
Geometiria jest jedną z najstarszych dziedzin matematyki.
Wybierz rozdzial ze spisu treści ponizej.

\sphinxstepscope


\chapter{\sphinxstylestrong{1. Figury płaskie}}
\label{\detokenize{rozdzial1:figury-plaskie}}\label{\detokenize{rozdzial1::doc}}
\sphinxAtStartPar
Figury płaskie to takie, które można narysować na kartce papieru, czyli
mają tylko dwa wymiary \textendash{} długość i szerokość. Do najważniejszych należą:
\begin{itemize}
\item {} 
\sphinxAtStartPar
\sphinxstylestrong{Trójkąt} \textendash{} figura o trzech bokach i trzech kątach; suma jego kątów
zawsze wynosi 180°.

\item {} 
\sphinxAtStartPar
\sphinxstylestrong{Kwadrat} \textendash{} ma cztery równe boki i cztery kąty proste.

\item {} 
\sphinxAtStartPar
\sphinxstylestrong{Prostokąt} \textendash{} przeciwległe boki są równe, a wszystkie kąty proste.

\item {} 
\sphinxAtStartPar
\sphinxstylestrong{Koło} \textendash{} zbiór wszystkich punktów na płaszczyźnie w tej samej
odległości od środka.

\end{itemize}

\sphinxAtStartPar
\sphinxincludegraphics[width=4.35417in,height=2.90278in]{{Obraz1}.png}

\sphinxstepscope


\chapter{\sphinxstylestrong{2. Figury przestrzenne}}
\label{\detokenize{rozdzial2:figury-przestrzenne}}\label{\detokenize{rozdzial2::doc}}
\sphinxAtStartPar
Figury przestrzenne (bryły) mają trzy wymiary: długość, szerokość i
wysokość. Ich objętość i pole powierzchni są kluczowe w wielu
zastosowaniach. Do najczęściej spotykanych należą:


\begin{savenotes}\sphinxattablestart
\sphinxthistablewithglobalstyle
\centering
\begin{tabular}[t]{*{4}{\X{1}{4}}}
\sphinxtoprule
\sphinxstyletheadfamily &\sphinxstyletheadfamily \begin{itemize}
\item {} 
\end{itemize}

\sphinxAtStartPar
\sphinxstyleemphasis{1.*}
&\sphinxstyletheadfamily 
\sphinxAtStartPar
\sphinxstylestrong{Sześcian}
&\sphinxstyletheadfamily 
\sphinxAtStartPar
wszystkie ściany są
kwadratami, a
krawędzie mają równą
długość.
\\
\sphinxmidrule
\sphinxtableatstartofbodyhook&\begin{itemize}
\item {} 
\end{itemize}

\sphinxAtStartPar
\sphinxstyleemphasis{2.*}
&
\sphinxAtStartPar
\sphinxstylestrong{Prostopadłościan}
&
\sphinxAtStartPar
ma sześć ścian
prostokątnych, często
używany do opisu
pudełek lub budynków.
\\
\sphinxhline&\begin{itemize}
\item {} 
\end{itemize}

\sphinxAtStartPar
\sphinxstyleemphasis{3.*}
&
\sphinxAtStartPar
\sphinxstylestrong{Kula}
&
\sphinxAtStartPar
zbiór wszystkich
punktów w przestrzeni
równo oddalonych od
środka.
\\
\sphinxhline&\begin{itemize}
\item {} 
\end{itemize}

\sphinxAtStartPar
\sphinxstyleemphasis{4.*}
&
\sphinxAtStartPar
\sphinxstylestrong{Stożek i walec}
&
\sphinxAtStartPar
mają podstawy w
kształcie koła, różnią
się jednak sposobem
łączenia z
wierzchołkiem lub
drugą podstawą.
\\
\sphinxbottomrule
\end{tabular}
\sphinxtableafterendhook\par
\sphinxattableend\end{savenotes}

\sphinxstepscope


\chapter{\sphinxstylestrong{3. Zastosowanie figur w życiu codziennym}}
\label{\detokenize{rozdzial3:zastosowanie-figur-w-zyciu-codziennym}}\label{\detokenize{rozdzial3::doc}}
\begin{DUlineblock}{0em}
\item[] Figury geometryczne są wszędzie wokół nas. W architekturze budynki
często opierają się na prostokątach i walcach, a w sztuce i
projektowaniu wykorzystuje się symetrię oraz proporcje geometryczne. W
informatyce geometria stanowi podstawę grafiki komputerowej,
modelowania 3D i projektowania gier.
\item[] Również w przyrodzie można zauważyć geometryczne wzory — struktura
plastra miodu przypomina sześciokąty, a planety mają kształty zbliżone
do kul.
\end{DUlineblock}



\renewcommand{\indexname}{Indeks}
\printindex
\end{document}